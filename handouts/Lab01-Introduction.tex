\documentclass[12pt]{scrartcl}

\setlength{\parindent}{0pt}
\setlength{\parskip}{.25cm}

\usepackage{graphicx}

\usepackage{xcolor}

\definecolor{darkred}{rgb}{0.5,0,0}
\definecolor{darkgreen}{rgb}{0,0.5,0}
\usepackage{hyperref}
\hypersetup{
  letterpaper,
  colorlinks,
  linkcolor=red,
  citecolor=darkgreen,
  menucolor=darkred,
  urlcolor=blue,
  pdfpagemode=none,
  pdftitle={Introduction To Git},
  pdfauthor={Christopher M. Bourke},
  pdfcreator={$ $Id: cv-us.tex,v 1.28 2009/01/01 00:00:00 cbourke Exp $ $},
  pdfsubject={PhD Thesis},
  pdfkeywords={}
}

\definecolor{MyDarkBlue}{rgb}{0,0.08,0.45}
\definecolor{MyDarkRed}{rgb}{0.45,0.08,0}
\definecolor{MyDarkGreen}{rgb}{0.08,0.45,0.08}

\definecolor{mintedBackground}{rgb}{0.95,0.95,0.95}
\definecolor{mintedInlineBackground}{rgb}{.90,.90,1}

%\usepackage{newfloat}
\usepackage[newfloat=true]{minted}
\setminted{mathescape,
               linenos,
               autogobble,
               frame=none,
               framesep=2mm,
               framerule=0.4pt,
               %label=foo,
               xleftmargin=2em,
               xrightmargin=0em,
               startinline=true,  %PHP only, allow it to omit the PHP Tags *** with this option, variables using dollar sign in comments are treated as latex math
               numbersep=10pt, %gap between line numbers and start of line
               style=default, %syntax highlighting style, default is "default"
               			    %gallery: http://help.farbox.com/pygments.html
			    	    %list available: pygmentize -L styles
               bgcolor=mintedBackground} %prevents breaking across pages
               
\setmintedinline{bgcolor={mintedBackground}}
\setminted[text]{bgcolor={mintedBackground},linenos=false,autogobble,xleftmargin=1em}
%\setminted[php]{bgcolor=mintedBackgroundPHP} %startinline=True}
\SetupFloatingEnvironment{listing}{name=Code Sample}
\SetupFloatingEnvironment{listing}{listname=List of Code Samples}

\title{CSCE 155 - C}
\subtitle{Lab 01 - Introduction}
\author{Dr.\ Chris Bourke}
\date{~}

\begin{document}

\maketitle

\section*{Prior to Lab}

In each lab there may be pre-lab activities that you are \emph{required} to
complete prior to attending lab.  Failure to do so may mean that you will
not be given full credit for the lab.  For the first lab, there are no pre-lab
activities.

\section*{Peer Programming Pair-Up}

To encourage collaboration and a team environment, labs will be
structured in a \emph{pair programming} setup.  At the start of
each lab, you will be randomly paired up with another student 
(conflicts such as absences will be dealt with by the lab instructor).
One of you will be designated the \emph{driver} and the other
the \emph{navigator}.  

The navigator will be responsible for reading the instructions and
telling the driver what to do next.  The driver will be in charge of the
keyboard and workstation.  Both driver and navigator are responsible
for suggesting fixes and solutions together.  Neither the navigator
nor the driver is ``in charge.''  Beyond your immediate pairing, you
are encouraged to help and interact and with other pairs in the lab.

Each week you should alternate: if you were a driver last week, 
be a navigator next, etc.  Resolve any issues (you were both drivers
last week) within your pair.  Ask the lab instructor to resolve issues
only when you cannot come to a consensus.  

Because of the peer programming setup of labs, it is absolutely 
essential that you complete any pre-lab activities and familiarize
yourself with the handouts prior to coming to lab.  Failure to do
so will negatively impact your ability to collaborate and work with 
others which may mean that you will not be able to complete the
lab.  

\section{Lab Objectives \& Topics}
At the end of this lab you should be familiar with the following
\begin{itemize}
  \item The general lab environment \& CSE system including its policies and expectations
  \item Basic unix commands
  \item Retrieving lab code from Github
  \item Modifying, compiling and executing your first C program
  \item Using CSE's web handin and web grader
\end{itemize}

\section{Activities}

\subsection{Login \& Consent Form}

You will receive your login from the CSE System Administrators who will also 
give you an overview of the CSE system and its policies.  Be sure to login and 
change your temporary password.  Some departmental resources that you may
find useful:
\begin{itemize}
  \item CSE Website: \url{http://cse.unl.edu}
  \item UNL Computing Policy: \url{http://www.unl.edu/ucomm/compuse/}
   \item CSE Academic Integrity Policy: \url{http://cse.unl.edu/academic-integrity-policy}
   \item CSE System FAQ: \url{http://cse.unl.edu/faq}
   \item Account Management: \url{https://cse-apps.unl.edu/amu/}
   \item CSE Undergraduate Advising Page: \url{http://cse.unl.edu/advising}
   \item CSE Student Resource Center: \url{http://cse.unl.edu/src}
\end{itemize}

\subsection{Lab Introduction}

\begin{itemize}
  \item Lab instructors and TAs
  \item Office Hours
  \item Student Resource Center (Avery 12)
  \item Lab policies
\end{itemize}

\subsection{Basic Unix Commands}

Much of your work will be done on the command line.  Take a moment to 
familiarize yourself with the following basic commands.  A more comprehensive 
tutorial on unix commands is available here: \url{http://www.math.utah.edu/lab/unix/unix-commands.html}.
\begin{itemize}
  \item Show the current working directory: \mintinline{text}{pwd}
  \item Creating a new directory: \mintinline{text}{mkdir dirName}
  \item Changing directories: \mintinline{text}{cd dir_name}
  \item Moving up a directory: \mintinline{text}{cd ..}
  \item Moving directly to your home directory: \mintinline{text}{cd ~}
  \item Listing files in a directory: \mintinline{text}{ls}
  \item Listing details of files in a directory: \mintinline{text}{ls -l}
  \item Listing files in another directory: \mintinline{text}{ls dir_name}
  \item Listing the contents of a (text) file: \mintinline{text}{more file_name}
  \item Removing files (careful!):  \mintinline{text}{rm file_name}
\end{itemize}

\subsubsection*{Text Editors}

Programming requires that you write code in a source file: a plain text file that 
contains syntactically valid programming commands.  In general, any plain 
text editor can facilitate this (MS Word is not a plain text editor).  The lab 
instructor will demonstrate some of the following options: 
\begin{itemize}
  \item jpico
  \item emacs
  \item Notepad++ (a graphical editor launched from Windows)
  \item Atom.io \url{https://atom.io/}
  \item Sublime Text \url{https://www.sublimetext.com/}
\end{itemize}

\subsection{Checking Out Code From Github}

Each lab will have some starter code and other \emph{artifacts} (data files, 
scripts, etc.) that will be provided for to you.  However, the code is hosted
on Github (\url{https://github.com}) and you must check it out.  You will not
need to know the details of using git nor be a registered Github user to 
get access to the code necessary for your labs or assignments.  However, 
you are \emph{highly encouraged} to learn this essential tool.  You may find
it very useful to keep track of your own code and to share code if you work
in pairs or groups.  

To check out the code for this lab, do the following.
\begin{enumerate}
  \item If you haven't already, connect to the CSE server using PuTTY as follows:
  \begin{enumerate}
    \item Open PuTTY (this is a Secure Shell Client, SSH)
    \item Enter the address \mintinline{text}{cse.unl.edu}
    \item Click ``Open''
    \item Login using your CSE credentials (you may be prompted to change your password the first time).
  \end{enumerate}
  \item In your home directory, create a new directory for all your labs:\\  
  \mintinline{text}{mkdir labs}  
  \item Verify that this worked by listing the contents of your directory by
  typing \mintinline{text}{ls}.  You should see your \mintinline{text}{labs} directory.
   \item Go to this directory by changing your directory:\\
  \mintinline{text}{cd labs}
  \item ``Clone'' this lab's code by using the following command:\\
  \mintinline{text}{git clone https://github.com/cbourke/CSCE155-C-Lab01}
  \item A new directory, \mintinline{text}{CSCE155-C-Lab01} should be created, 
  go to this directory by typing \mintinline{text}{cd CSCE155-C-Lab01}.
  \item List the contents of this directory by typing \mintinline{text}{ls}.  If
  everything worked, there should be a \mintinline{text}{hello.c} file in your
  directory.
\end{enumerate}

\subsection{Your First Program: Hello World!}

As this is an introductory course, most of the C programs you will write in 
this course will execute and require interaction from the command line 
rather than a more user-friendly graphical interface.  

The code you cloned contains a basic ``Hello World'' program that, when
compiled and run will simply print the message ``Hello World'' and exit.

To compile and run this program, do the following.
\begin{enumerate}
  \item Type \mintinline{text}{gcc hello.c}, this \emph{compiles} the source code
  	into an \emph{executable} file named \mintinline{text}{a.out}.  Use 
	\mintinline{text}{ls} to verify that the new file has been created.  (Note: 
	\mintinline{text}{a.out} is the default file name, to specify a different
	executable file name you can instead use \mintinline{text}{gcc -o foo hello.c}
	for example, to compile to a file named \mintinline{text}{foo} instead of
	\mintinline{text}{a.out}).
  \item Run your program by typing the following: \mintinline{text}{./a.out}
\end{enumerate}

Let's now modify the program. 

\begin{enumerate}
  \item Open the \mintinline{text}{hello.c} source file in the text editor of
  	your choice (Notepad++ is recommended for beginners).
  	Your program should look something like the following:

\begin{minted}{c}
/**
 * Author: Chris Bourke
 * Date: 2016/11/02
 *
 * A simple hello world program in C
 *
 */
#include<stdlib.h>
#include<stdio.h>

int main(int argc, char **argv) {

  printf("Hello World!\n");

  return 0;
}
\end{minted}
  \item Modify the file by changing the author to you and your partner and change
  	the date.
  \item Change the message that is produced by the program (line 10) to instead
  	print you and your partner's names.
  \item Save and exit your editor and repeat the process to compile and run
  	your program and verify that your changes have taken effect.
\end{enumerate}

\section{Handing In \& Grading}

Many of your assignments will include a programming portion that will 
require you to hand in \emph{soft-copy} source files for graders to compile 
and evaluate.  To do this, you will use a web-based handin program.  
After handing your file(s) in, you can then grade them by using the
web grader.  To demonstrate, do the following.

\begin{enumerate}
  \item Open a browser to \url{https://cse-apps.unl.edu/handin}
  \item Login with your CSE credentials
  \item Click on this course/lab 01 and handin the \mintinline{text}{hello.c} file.  You can
  either click the large ``handin'' area and select the file or you can drag-drop
  the file.  You will be able to re-handin the same file as many times as you want up until
  the due date.
  \item Now that the file has been handed in, you can ``grade'' yourself by using
  the webgrader; open a new tab/window and point your browser to one of the following
  depending on which course you are in:
  \begin{itemize}
    \item \url{https://cse.unl.edu/~cse155a/grade}
    \item \url{https://cse.unl.edu/~cse155e/grade}
    \item \url{https://cse.unl.edu/~cse155h/grade}
  \end{itemize}
  \item Fill the form with your CSE login and password, select the appropriate assignment (Lab 01)
  	and click ``Grade''
  \item For future assignments and labs, you can compare the results of 
  	your program with the ``Expected Results''.  If there are problems or errors with
	your program(s), you should fix/debug them and repeat the handin/grading process.
	You can do this as many times as you like up until the due date.  Some programs 
	and assignments will run test cases and may provide expected output alongside 
	your output.  Others may have more sophisticated test cases and actually provide 
	you a percentage of test cases passed.  It is your responsibility to read, 
	understand and \emph{address} all of the errors and/or warnings that the grader produces.
\end{enumerate}

Demonstrate your working programs to a lab instructor, answer all the 
questions on your lab worksheet, and have them sign off on it.

\section*{Resources}

\begin{itemize} 
  \item Windows users can download and use PuTTY on their own machines: \\
  	\url{http://www.chiark.greenend.org.uk/~sgtatham/putty/download.html}
  \item Mac users already have an SSH client as part of their OS.  Open a terminal and type:
  	\mintinline{text}{ssh login@cse.unl.edu} where login is replaced with
	your own login; provide your password and you're at the CSE command line
  \item You may also find FileZilla useful (\url{http://filezilla-project.org/}).  It is a 
  	free File Transfer Protocol (FTP) client that you can use to transfer files 
	back and forth from your machine to your CSE account.
  \item Many other options exist for developing code on your own machine, see the 
    course page for more resources.
\end{itemize}

\end{document}
